\documentclass{article}
\begin{document}
\title{IUM}

\maketitle

\tableofcontents

\section{Applied}

\subsection{From Newton to the moon}

Ordered pairs: \math (a,b) = (a',b') \Leftrightarrow (a = a' \wedge b = b')\)
\newline Cartesian product \math A \times B = {(a,b) \vert a \in A, b \in B}, A^2\)
\\
Cartesian coordinates: \math (x,y,z) \in R^3\)
\newline Polar coordinates: \math (r,\phi) \in R^2, r \geq 0, \phi (-\pi, \pi]\)
\newline Cylindrical coordinates: \math (\rho,\phi,z) \in R^3, \rho > 0, \phi \in (-\pi,\pi]\)
\newline Spherical coordinates: \math (r,\phi,\theta) \in R^3, r \geq 0, \phi \in (-\pi,\pi], \theta \in [0,\pi]\)
\\
\\
Implicit definition of curve or surface as a set of points that fulfil a set of constraints. Parametric definition
\\
\\
Lines in R^2\):

- \left\{ \begin{array}{rcl}\math x = $x_0$+\lambda & \mbox{with} & \lambda \in R \\ y = $y_0$+\lambda \alpha \end{array}\right
\)

- \math ax + by + c = 0\) with \math (a,b) \neq (0,0)\)

- Goes through pole: \theta = $\theta_0$\) with $\theta_0$ \in R\)

- Doesn't go through pole: \math r = \frac{1}{\alpha cos(\theta)+\beta cos(\theta)}\) with \math (\alpha,\beta) \neq (0,0)\)
\\
\\
Planes and lines in R^3\):

- \left\{ \begin{array}{rcl}\math x = $x_0$+\lambda $u_1$+\mu $v_1$ & \mbox{with} & \lambda \in R \\ y = $y_0$ + \lambda $u_2$ + \mu $v_2$ \\ z = $z_0$ + \lambda $u_3$ + \mu $v_3$ \end{array}\right\)

- \math ax + by + cz + d = 0\), where \math (a,b,c) \neq (0,0,0)\) defines planes in R^3\)

- \left\{ \begin{array}{rcl}\math x = $x_0$+\lambda $u_1$ & \mbox{with} & \lambda \in R \\ y = $y_0$ + \lambda $u_2$ \\ z = $z_0$ + \lambda $u_3$ \end{array}\right\)

- \math $a_1$x + $b_1$y + $c_1$z + $d_1$ = 0

\newline $a_2$x + $b_2$y + $c_2$z + $d_2$ = 0\)
\\
\\
Given a point (focus) and a line not containing this point (directrix), set of points for which the ratio of the distances to the focus and the directrix is e \in R^>\) (eccentricity)

- Ellipse when \math e < 1\)

- Parabola when \math e = 1\)

- Hyperbola when \math e > 1\)

Special case for which \math e = 0\) (directrix at infinity) is a circle
\newline Cartesian equation of conic section: \math Ax^2 + Bxy + Cy^2 + Dx + Ey + F = 0\), where A,B,C,D,E,F \in R \triangle := B^2 - 4AC\)

- Ellipse for \triangle < 0\)

- Parabola for \triangle = 0\)

- Hyperbola for \triangle > 0\)
\newline Polar equation of conic section with one focus at the pole: \math r = \frac{l}{1+e cos(\phi+$\phi_0$)}\) where \math e,l \in R, e\) is eccentricity, \math l\) is semi-latus rectus (length of straight line between focus and conic section along direction parallel to directrix)
\newline Circular cone: \math x^2 + y^2 - k^2z^2 = 0\)
\\
\\
Kepler's first law: The planets orbit around the sun on elliptical orbits with the sun in one of the focal points.
\newline Kepler's second law: The line connecting the sun and the sun and the planet sweeps equal areas of space during equal time intervals.
\newline Kepler's third law: The squares of the orbital periods of the planets are proportional to the cubes of the semi-major axes of their orbits.
\\
\\
Position: \math (x,y,z) \in R^3\)
\newline Velocity: \math ($v_x$,$v_y$,$v_z$) = (\.{x},\.{y},\.{z})\)
\newline Momentum: \math $p_j$ = m$v_j$
\newline Solution \math (x(t),y(t),z(t))\) is called trajectory of particle
\newline Newton's second law in vector form: \.{p} = F\), where F is external force on "body"/"particle"
\\
\\
Rotations in the plane are commutative and amount to one, multiplying complex numbers in polar form
\\
\\
\math i^2 = j^2 = k^2 = ijk = -1 \)
\newline A quaternion q is then an object of the form \math q = $q_0$ + i$q_1$ + j$q_2$ + k$q_3$\) with \math $q_0,1,2,3$ \in R\)
\\
\\
\vert u \vert = sqrt{$u_1$^2,$u_2$^2,$u_3$^2}\) (euclidean norm, find distance between 2 points)
\newline "normalised", \vert u \vert = 1 = \hat{u}\) (unit vector)
\newline \math(0,0,0)\) (zero vector)
\newline Scalar multiplication and vector addition/subtraction (commutative, associative and distributive)
\\
\\
Parallelogram rule: \vert u + v \vert^2 + \vert u - v \vert^2 = 2 \vert u \vert^2 + 2 \vert v \vert^2\)
\\
\\
A set of N vectors {$u_j$} is called linearly independent if no non-trivial linear combination (not all coefficients are 0) of them sums to 0. On the contrary, called linearly dependent if at least one of the vectors in the set as a linear combination of the others: \math $u_k$ = \sum_{j \neq k}$a_j$$u_j$\)
\\
\\
A set of three vectors a, b, c is a basis of R^3\), if any vector \math u \in R^3\) can be expressed as a unique linear superposition of a, b and c: \math u = $u_a$a + $u_b$b + $u_c$c\) (N vectors going in different directions are able to go to any point in N-dimensions)

Standard basis of R^3: \math (1,0,0),(0,1,0),(0,0,1) \hat{$e_1$} \hat{$e_2$} \hat{$e_3$}\)
\\
\\
\math Det: (R^n\))^n \rightarrow R\), fulfils following properties:

- Linearity in each argument 

- Antisymmetry with respect to interchange of arguments
\newline To make determinant uniquely defined, pick a basis {$u_j$} and set \math det($u_1$,...,$u_n$) = 1\)
\newline (Think about the positive cycle of \math 1 \rightarrow 2 \rightarrow 3 \rightarrow 1\) to determine signs)
\newline \math det(u,v) = $u_1$$v_2$ - $u_2$$v_1$\)
\newline \math det(u,v,w) = $u_1$$v_2$$w_3$ + $u_2$$v_3$$w_1$ + $u_3$$v_1$$w_2$ - $u_3$$v_2$$w_1$ - $u_2$$v_1$$w_3$ - $u_1$$v_3$$w_2$\)
\\
\\
The determinant of \math n\) non-zero vectors is zero if and only if the vectors are linearly dependent
\\
\\
Dot/scalar product in R^3\): (Symmetric bilinear form)

\math u \cdot v = \vert u \vert \vert v \vert cos(\phi) = $u_1$$v_1$ + $u_2$$v_2$ + $u_3$$v_3$\)
\\
\\
Cauchy-Schwarz Inequality: let u and v be 2 vectors in R^3\), then:

\math \vert u \cdot v \vert \leq \vert u \vert \vert v \vert\), with equality when u and v are linearly dependent
\newline Triangle Inequality: let u and v be 2 vectors in R^3\), then:

\math \vert u + v \vert \leq \vert u \vert + \vert v \vert\)
\\
\\
Let u and v be vectors in R^3\) with \math v \neq 0\), there is a unique real \lambda\) such that \math u - \lambda v\) is perpendicular to v: \math u = \lambda v + (u - \lambda v)

\lambda = \frac{u \cdot v}{\vert v \vert^2}

$proj_v$u = \frac{u \cdot v}{\vert v \vert^2} v\)
\\
\\
A set of vectors {$u_j$} is said to be orthonormal if each vector is normalised and pairwise orthogonal

\math $u_j$ \cdot $u_k$ = $\delta_jk$ \)
\newline Kronecker symbol $\delta_jk$ = \left\{ \begin{arrary}{rcl} 1 & if j=k \\ 0 & otherwise \end{array}\right\)
\\
\\
\math u = \alpha $\hat{e}_1$+\beta $\hat{e}_2$, v = -\beta$\hat{e}_1$+\alpha $\hat{e}_2$\) orthogonal to \^{u}, \lambda \^{v}\) vectors in R^2\) orthogonal to \^{u}, \lambda \in R\)
\\
\\
Plane P oriented by orthonormal basis {$\hat{e}_1$, $\hat{e}_2$}, an orthonormal basis {\^{u},\^{v}\)} is said to be right-handed if vectors are given by:
\newline \^{u} = \alpha $\hat{e}_1$+\beta $\hat{e}_2$\) and \^{v} = -\beta $\hat{e}_1$+\alpha $\hat{e}_2$\)
\newline and left_handed by: \^{u} = \alpha $\hat{e}_1$+\beta $\hat{e}_2$\) and \^{v} = \beta $\hat{e}_1$-\alpha $\hat{e}_2$\)
\newline with \math (\alpha,\beta) \in R^2\) and \alpha^2 + \beta^2 = 1\)
\\
\\
Vector product: (Antisymmetric bilinear form)

\newline \math (($y_1$$z_2$ - $y_2$$z_1$),($z_1$$x_2$ - $z_2$$x_1$),($x_1$$y_2$ - $x_2$$y_1$))\)
\\
\\
\math u \times v = det(u,v) = \vert u \vert \vert v \vert sin(\phi)\) (Area of parallelogram spanned by the two vectors u and v)
\newline \vert u \times v \vert^2 = \vert u \vert^2 \vert v \vert^2 - (u \cdot v)^2\)
\\
\\
Let {\^{u},\^{v},\^{w}\)} an orthonormal basis: right-handed if \^{w} = \^{u} \times \^{v}\) and left-handed if \^{w} = -\^{u} \times \^{v}\)
\newline (first finger: u, second finger: v, thumb: u x v)
\\
\\
Scalar triple product: \math [u,v,w] = u \cdot (v \times w) = det(u,v,w)\)
\newline Volume of parallelepiped \math \vert [u,v,w] \vert
\newline [u,v,w] = [v,w,u] = [w,u,v] = -[u,w,v] = -[w,v,u] = -[v,u,w] \)
\\
\\
Vector triple product: \math u \times (v \times w)
\newline u \times (v \times w) = (u \cdot w)v - (u \cdot v)w\)

\subsection{Vector spaces and atomic physics}

(Abelian) Group, set of elements together with a binary operation: (map from \math G \times G \rightarrow G, \circ\))

- Closure: if a and b are elements in G then a \circ\) b is also an element of G

- Associativity: \forall a,b,c \in G\) it holds \math (a \circ b) \circ c = a \circ (b \circ c)\)

- Identity element: \math I \in G\), such that \math a \circ I = I \circ a = a, \forall a \in G\)

- Inverse \forall a \in G, \exists b \in G\), the inverse of a, such that \math a \circ b = b \circ a = I\)

- (Abelian) Commutativity: \forall a,b \in G, a \circ b = b \circ a\)
\\
\\
Field, set with two commutative operations
\\
\\
Vector space, let F be a field, the elements of which we will denote as scalars. A set of elements V, forming an Abelian group under a vector addition V \times\) V \rightarrow\) V (Vector space over the field F): \lambda,\mu \in\) F and \math u,v \in\) V

- Distributivity: \math (\lambda + \mu)v = \lambda v + \mu v\) and \math \lambda (u + v) = \lambda u + \lambda v\)

- Associativity: (\lambda \mu)v = \lambda (\mu v)

- \math 1v = v\), where 1 denotes the identity element of the multiplication in F

Elements of the vector space V are called vectors. The identity element of the vector addition is called the zero vector
\\
\\
Sub vector space, non-empty subset W \subset\) V of a vector space is a sub vector space of V if:

- \math $w_1$ + $w_2$ \in W, \forall $w_1$,$w_2$ \in W\)

- \lambda w \in W, \forall w \in W\), and \lambda \in\) F
\\
\\
Set of all linear combinations of a given set of vectors {$u_j$} with arbitrary coefficients is called the span of the vectors {$u_j$}, is a sub vector space of V
\\
\\
Largest possible number N of linearly independent vectors in a vector space V is called the dimension of V
\\
\\
Finite-dimensional vector space V over the field F, map \math T:V \rightarrow V\) is linear if \math T(\lambda u + \mu v) = \lambda T(u) + \mu T(v) \forall \lambda,\mu \in F, u,v \in V\)
\newline Let {$v_j$} be the basis of the vector space V. A linear map \math T:V \rightarrow V\) is uniquely defined by the set of vectors {T($v_j$)}
\newline There exist a unique set of scalars such that they form a matrix of the linear map (Vector in columns) 
\newline It is invertible if there exist a linear map \math T^-1\) such that \math T^-1(T(v)) = v = T(T^-1(v)), \forall v \in V\). If the determinant of its matrix representation is non-zero.
\\
\\
(Complex) Inner product space is a complex vector space V together with a map V \times\) V \rightarrow\) C:

- (Anti)-symmetry: \math (u,v) = (v,u)^*\) (asterisk denotes complex conjugation)

- Non-negativity: \math (u,u) \geq 0, (u,u) = 0 \Leftrightarrow u = 0\)

- Linearity in one of the elements (The other is the complex conjugate): \math (u,\lambda v + \mu v) = \lambda (u,v) + \mu (u,w)\)
\\
\\
Norm \vert v \vert^2 = (v,v)\)
\newline Cauchy-Schwarz inequality:

\vert u \cdot v \vert \leq \vert u \vert \vert v \vert\)
\newline Triangle inequality:

\vert u + v \vert \leq \vert u \vert + \vert v \vert\)
\\
\\
Orthogonal if \math (u,v) = 0 = (v,u)\)
\newline Orthonormal if each vector is normalised and pairwise orthogonal \math ($u_j$,$u_k$) = $\delta_jk$ \)
\newline Coefficients of a vector can be found by orthonormality condition \math $u_j$ = ($b_j$,u)\)
\\
\\
Let \math u = \sum_{j=1}^{N} $u_j$ $b_j$, v = \sum_{j=1}^{N} $v_j$ $b_j$\)

Their inner product is: \math (u,v) = \sum_{j=1}^{N} $u_j$^* $v_j$\)
\\
\\
Standard/canonical basis

\section{Logic and Sets}

\subsection{Sets and Logic}

A logical proposition is a true/false statement
\newline "And" \wedge\)
\newline "Or" \vee\)
\newline "Not" \neg\)
\newline "Implies" \Rightarrow\) (Only way to disprove if something true implies something false)
\newline "If and only if" \Leftrightarrow\) (Prove both ways)
\newline Propositional formula are logically equivalent if they have the same truth table, propositions are logically equivalent if and only if they are equal
\\
\\
Law of excluded middle: If P is a proposition, then \neg\) (\neg\) P) \Leftrightarrow\) P
\newline De Morgan's laws:

- \neg\) (P \vee\) Q) \Leftrightarrow\) (\neg\) P) \wedge\) (\neg\) Q)

- \neg\) (P \wedge\) Q) \Leftrightarrow\) (\neg\) P) \vee\) (\neg\) Q)
\newline Transitivity: if P \Rightarrow\) Q and Q \Rightarrow\) R, then P \Rightarrow\) R
\newline Contrapositive: (P \Rightarrow\) Q) \Leftrightarrow\) (\neg\) Q \Rightarrow \neg\) P)
\newline Distributivity:

- P \wedge\) (Q \vee\) R) \Leftrightarrow\) (P \wedge\) Q) \vee\) (P \wedge\) R)

- P \vee\) (Q \wedge\) R) \Leftrightarrow\) (P \vee\) Q) \wedge\) (P \vee\) R)
\\
\\
Square root of 2 is irrational:

Lemma: If n is an integer, then n is even if and only if n^2 is even

Use contradiction to prove that there is no rational number whose square is 2
\\
\\
Sets:

- Can only have elements in once

- Elements don't come in a particular order

- X \diagdown\) Y "What you get after you remove Y from X"

- \={X \cup Y} = \={X} \cap \={Y}

- \={X \cap Y} = \={X} \cup \={Y}\)
\\
\\
- \neg (\forall x \in X, P(x)) \Leftrightarrow \exists x \in X, \neg (P(x))

- \neg (\exists x \in X, P(x)) \Leftrightarrow \forall x \in X, \neg (P(x))\)
\\
\\
\exists x \in \varnothing\), P is always false (if P were true, would imply such an x exists but it doesn't)
\newline \forall x \in \varnothing\), P is always true (it is the logical negation, \exists x \in \varnothing, \neg P\))

\subsection{Functions and Equivalence Relations}

\math f:X \rightarrow Y\) is a function if: (domain and codomain)

- When you give \math x \in X\), it will return an element (not multiple)

- Function depends on domain and codomain

- Value of f(x) can't change over time
\\
\\
Image/Range of \math f:{y \in Y \vert \exists x \in X, f(x) = y}\)
\\
\\
Identity function: if x \in X, x \mapsto  x\)
\newline Empty function: function from empty set X to Y
\\
\\
\math f:A \rightarrow B\) and \math g:B \rightarrow C\), then \math g \circ f:A \rightarrow C\), (g composed with f)
\newline \math $f_1$:X \rightarrow Y\) and \math $f_2$:X \rightarrow Y\), they are equal if, \forall x \in X\) we have \math $f_1$(x) = $f_2$(x)\)
\\
\\
Injective: if \forall a,b \in X, f(a) = f(b) \Rightarrow a=b\)
\newline Surjective: if \forall y \in Y, \exists x \in X, f(x) = y\)
\newline Bijective: if it is both injective and surjective
\\
\\
If f and g are both injective/surjective/bijective, then so is \math g \circ f\)
\\
\\
Inverse: say \math f:X \rightarrow Y\) is a function, \math g:Y \rightarrow X\) is a two-sided inverse of f if:
\newline \forall x \in X, g(f(x)) = x
\newline \forall y \in Y, f(g(y)) = y
\newline \math f:X \rightarrow Y\) has a two-sided inverse if and only if it's a bijection,
\\
\\
A binary relation R on a set X is a subset of X^2\), this set is denoted by \math R(x,y)\)
\\
\\
Reflexive: \forall x \in X\), \math R(x,x)\) is true
\newline Symmetric: \forall a,b \in X, R(a,b) \Rightarrow R(b,a)\)
\newline Antisymmetric: \forall a,b \in X, R(a,b) \wedge R(b,a) \Rightarrow a = b\)
\newline Transitive: for every \math a,b,c \in X,\) we have \math (R(a,b) \wedge R(b,c)) \Rightarrow R(a,c)\)

- R is a partial order if it's reflexive, antisymmetric and transitive
- R is total if for all \math a,b \in X\), either \math R(a,b) \vee R(b,a)\) is true
- R is a total order if it's a partial order and total
\\
\\
R is an equivalence relation if it is reflexive, antisymmetric and transitive
\\
\\
Let X be a set and let ~ be an equivalence relation on X. Let s \in\) X be an arbitrary element We define the equivalence class of S, written cl(s), to be the set of elements of X which s is related to. \math cl(s) = {x \in X: s~x}\)

- If s∼t then \math cl(t) \subset cl(s) \Rightarrow cl(t) = cl(s)\)

- if \neg (s∼t)\) then \math cl(t) \cap cl(s) = \emptyset\)
\\
\\
Partition of a set X is a set A of non-empty subsets of X with the property that each element of X is in exactly one of the subsets (equivalence classes)

\section{Numbers}

\subsection{Natural Numbers}

Peano axioms:

- 0 is a natural number

- If n is a natural number, then its successor S(n) is a natural number (successor function telling you how to count, we haven't defined addition yet)

- That's it (essential for proving something for all natural numbers, that as 0 and S are the only way to make natural numbers)
\\
\\
Addition:

- \math a + 0 = 0\)

- \math a + S(n) = S(a + n)\)
\\
\\
Induction on n (fix other variables):

- (Base case) prove that $P_0$ is true

- (Inductive case) prove that if $P_n$ is true, then $P_S(n)$ is true
\newline Recursion: defining an object/function by referring a smaller version of itself
\newline (If you want to do something x times, you must use induction or recursion ("that's it") in a formal system)
\\
\\
Basic properties:

- \math 0 + x = x\)

- \math S(a) + x = S(a + x)\)

- \math a + b = b + a\) (Commutativity of addition)

- \math (a + b) + c = a + (b + c)\) (Associativity of addition)
\\
\\
Multiplication:

- \math a \times 0 = 0\)

- \math a \times S(n) = a \times n + a\)
\\
\\
\newline Basic properties:

- \math x \times 1 = x\)

- \math 0 \times x = 0\)

- \math S(x)y = xy + y\)

- \math xy = yx\) (Commutativity of multiplication)

- \math 1 \times x = x\)

- \math x(y + z) = xy + xz\) (Distributivity of multiplication)

- \math (x + y)z = xz + yz\)

- \math (xy)z = x(yz)\) (Associativity of multiplication)
\\
\\
Peano's remaining axioms (useful for proving "if some equation is true, then some other equation is also true"):

- If x is a natural number, then \math S(x) \neq 0\) (if you keep applying S() you will never end up back at 0)

- If x and y are natural numbers and \math S(x) = S(y)\), then \math x = y\) (\math S()\) is injective)
\\
\\
Other theorems:

- If \math x + n = y + n\) then \math x = y\)

- If \math x + n = n\), then \math x = 0\)

- \math x + y = 0\), then \math x = y = 0\)
\\
\\
Ordering on the naturals: We say that \math x \leq y\) if there exists a natural number n such that \math y = x + n\)

- \math 0 \leq x\)

- \math x \leq x\) (reflexivity)

- \math x \leq S(x)\)

- If \math x \leq 0\) then \math x = 0\)

- If \math x \leq y\) and \math y \leq z\) then \math x \leq z\) (transitivity)

- If \math x \leq y\) and \math y \leq x\) then \math x = y\) (antisymmetry)

- Either \math x \leq y\) or \math y \leq x\) (totality)

- If \math a \leq b\) then \math a + x \leq b + x\) 

- If \math a \leq b\) then \math a \times x \leq b \times x\)
\\
\\
Strong induction (no base case): For every n, you can deduce $P_n$, if you assume $P_t$ \forall t < n\)

- Deduce that the naturals are well-ordered, if it has one or more elements in a set then it has a smallest element
\newline Strong recursion: It is safe to define a function from N to a set X by a recursive formula in which F(n) depends on any or all previous values of F
\\
\\
Quotient function \math $q_b$:N \rightarrow N\)

- If \math a \geq b\): then there exists k such that \math a = b + k\), and we set \math $q_b$(a) = $q_b$(k) + 1\)

- Otherwise: we set \math $q_b$(a)\) to be 0
\newline Remainder function \math $r_b$:N \rightarrow N\)

- If \math a \geq b\): then there exists k such that \math a = b + k\), and we set \math $r_b$(a) = $r_b$(k)\)

- Otherwise: we set \math $r_b$(a)\) to be a
\newline Quotient-Remainder Theorem: (q is quotient and r is remainder when a is divided by b)

- \math 0 \leq r < b\)

- \math a = bq + r\)
\newline We say that \math m \vert n\) if there exists a number k such that \math n = m \times k\)

- If d divides both a and \math a + k\), then d divides k

- Prime number is \math n \geq 2\) which is divisible only by 1 and itself, otherwise composite

- Every \math n \geq 2\) can be factored as a finite product of prime numbers

- There are infinitely many primes
\\
\\
Euclid's algorithm: \math $gcd_a$:N \rightarrow N\) (gcd stands for greatest common divisor)

- \math $gcd_0$\) is the identity function: \forall b, $gcd_0$(b) = b\)

- For nonzero a, \math $gcd_a$(b) = $gcd_$r_a$(b)$(a)\)
\newline Results from this:

- \math gcd(a,b)\) is always positive, unless both inputs are 0

- \math gcd(a,b)\) divides both a and b

- If x is any number which divides a and b, then it also divides \math gcd(a,b)\)

\subsection{Integers}

We want to assign the integers as the difference between 2 natural numbers, but there are many ways we can do this. Therefore, we put an equivalence relation on the set, and then consider equivalence classes. Since we haven't defined subtraction, we can't relate pairs \math (a,b)\) and \math (c,d)\) with \math a - b = c - d\), instead we use \math a + d = b + c\).

Prove that \math (a,b)∼(c,d)\) if and only if \math a + d = b + c\) is an equivalence relation
\newline Set of integers is the set of all equivalence classes for N^2\) under this equivalence relation
\\
\\
We want N \subset\) Z, but the definition natural number 2 and integer 2 are not equal, Define function i from N to Z, sending natural number n to equivalence class of \math (n,0)\). (i is injective)
\\
\\
(pre-definition on N^2\), playing well with equivalence means if \math a = b\) then \math f(a) = f(b)\), show that it is well-defined)
\newline Subtraction:

- Pre-negation function, defining \math -(a,b)\) to be \math (b,a)\)

- If \math (a,b) $\sim$ (c,d)\) then \math -(a,b) ∼ -(c,d)\)

- Negation on integers, function sending equivalence class of \math (a,b)\) to the equivalence class of \math (b,a)\)

- Subtraction on integers, \math x - y = x + (-y)\)
\newline Addition:

- Pre-addition function, \math (a,b)\) and \math (c,d)\) return \math (a + c,b + d)\)

- If \math (a,b) $\sim$ (a',b')\) and \math (c,d) $\sim$ (c',d')\) then \math (a + c,b + d) $\sim$ (a' + c',b' + d')\)

- Addition on integers, \math cl(a,b) + cl(c,d) = cl(a + c,b + d)\)
\newline Multiplication:

- Pre-multiplication function, \math (a,b) \times (c,d) = (ac + bd,ad + bc)\)

- If \math (a,b) $\sim$ (a',b')\) then \math (a,b) \times (c,d) $\sim$ (a',b') \times (c,d)\)

- If \math (c,d) $\sim$ (c',d')\) then \math (a,b) \times (c,d) $\sim$ (a,b) \times (c',d')\)

- Multiplication on integers, \math cl(a,b) \times cl(c,d) = cl(ac + bd,ad + bc)\)
\\
\\
There are integers \lambda\) and \mu\) such that \math gcd(a,b) = \lambda a + \mu b\)
\newline If a and b are coprime, then there exists integers \lambda\) and \mu\) such that \lambda a + \mu b = 1\)
\newline Euclid's lemma: if p is prime and if a,b are natural numbers, and if p \vert\) ab, then p \vert\) a or p \vert\) b
\newline if p is prime and if $a_1$,...,$a_2$ are natural numbers, and if p \vert\) $a_1$,...,$a_n$ then p divides one of $a_i$
\newline Proving prime factorisation is unique: (strong induction)
\newline Quotient-remainder for integers: there exists integers q and r with \math 0 \leq r < b\) and \math a = qb + r\)
\newline If a is an integer, b is positive integer, and \math a = qb + r = q'b + r'\) with \math 0 \leq r\),\math r' < b\) then \math q = q'\) and \math r = r'\)
\\
\\
a \equiv\) b mod n, if n \vert (a - b)\)

- a \equiv\) b mod n

- If a \equiv\) b mod n, then b \equiv\) a mod n

- If a \equiv\) b mod n and b \equiv\) c mod n, then a \equiv\) c mod n

- a \equiv\) b mod n if and only if a and b have the same remainder after division by n

- There are exactly n congruence classes modulo n
\\
\\
Pre-addition and pre-multiplication for integers mod n:

- \math a + b \equiv a' + b'\) mod n

- \math ab \equiv a'b'\) mod n

\end{document}
