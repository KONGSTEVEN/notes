\documentclass{article}
\begin{document}
\title{IUM}

\maketitle

\tableofcontents

\section{Applied}

Ordered pair: \math (a,b) = (a',b') \Leftrightarrow (a=a' \wedge b=b')\)
\newline Set formed by ordered pairs \math (a,b)\) with a \in A\) and b \in B\) is called Cartesian product of A and B: A x B = \math \{(a,b)\vert a \in A, b \in B\}\)
\newline When A = B, we denote A x A by A^2\), extends to n-tuple
\\
\\
Cartesian coordinates: \math(x,y,z) \in \Re^3\)
\newline Distance between 2 points \math \triangle ($M_1$,$M_1$) = \sqrt{($x_1$-$x_2$)^2+($y_1$-$y_2$)^2+($z_1$-$z_2$)^2}\)
\\
\\
Polar coordinates: \math(r,\phi) \in \Re^2\)
\newline X-axis is polar axis, r is radial distance/radius, \phi\) is polar angle/azimuth, pole is the origin (any pair with r = 0)
\newline Therefore, ($r_0$,$\phi_0$) corresponds to r = $r_0$ and \phi = $\phi_0$ + 2n\pi\), n is arbitrary integer
\newline Possess uniqueness when \math r > 0\) and -\pi < \phi \leq \pi\), usually r \geq\) 0 with r \in \Re\) and \phi \in (-\pi,\pi]\)
\\
\\
Cylindrical coordinates: \math(\rho,\phi,z) \in \Re^3\)
\newline Spherical coordinates: \math(r,\phi,\theta) \in \Re^3\), such that r \geq 0, \phi \in (-\pi,\pi]\) and \theta \in (0,\pi]\)
\\
\\
An equation \math f(x,y,z) = 0\) is a cartesian equation of a part of \Re^3\) denoted by A if we have the equivalence: \math M(x,y,z) \in A \Leftrightarrow f(x,y,z) = 0 \)
\newline Many different cartesian equations can specify same subset A of \Re^3\)
\newline Every cartesian equation lowers dimensionality by one (unless constraints are related or are incompatible)
\\
\\
\math f(r,\phi) = 0\) in \Re^2\) is a polar equation of a part of a plane A if: a point M is a point of A if and only if one of its systems of polar coordinates verifies \math f(r,\phi) = 0
\\
\\
f(r,\phi,\theta) = 0\) is a spherical equation for a part A of \Re^3\) if: a point M is a point of A if and only if one of its systems of spherical coordinates verifies \math f(r,\phi,\theta) = 0\)
\\
\\
Definition of a curve/surface as a set of points that fulfil a set of constraints is called implicit definition
\newline Could have different parameterisations (e.g cartesian to polar coordinates)
\\
\\
2-dimensional subset of \Re^3\) is obtained by defining coordinates as functions of 2 parameters. Curve is obtained by expressing all coordinates as functions of one of the parameters. Can obtain a cartesian equation by eliminating the parameter(s).
\\
\\
For a given curve there are infinitely many possible parameterisations, if we view parameter t as time. Same curve could be traversed slower or faster in specific places.
\\
\\
Line \math L\) going through point P with cartesian ($x_0$,$y_0$) and slope \alpha\) can be parameterised as:
\left\{ \begin{array}{rcl}\math x = $x_0$+\lambda & \mbox{with} & \lambda \in \Re \\ y = $y_0$+\lambda \alpha \end{array}\right
\)
\newline Our parameterisation is the map \lambda \mapsto (x(\lambda), y(\lambda))\), the real \lambda\) is the parameter (changing to 2\lambda\) or \lambda^3\) changes the parameterisation)
\\
\\
Any line \math L\) in \Re^2\) can be represented as set of (x,y) fulfiling: \math ax + by + c = 0\) with (a,b) \neq (0,0)\)
\newline Any such equation represents a line, 2 such equations represent same line if they are proportional to each other.
\\
\\
Line that goes through pole defined by the polar equation: \theta = $\theta_0$ with $\theta_0$ \in \Re\)
\\
\\
Line that doesn't pass through pole: \math r = 1/(\alpha cos(\theta)+\beta cos(\theta))\) with (\alpha,\beta) \neq (0,0)
\newline \alpha \cdot r cos(\theta) + \beta \cdot r cos(\theta) = c\) (only pass through origin if c = 0, but then we can't obtain original result)
\\
\\
Set of points with cartesian \math (x,y,z)\) defined as:
\left\{ \begin{array}{rcl}\math x = $x_0$+\lambda $u_1$+\mu $v_1$ & \mbox{with} & \lambda \in \Re \\ y = $y_0$ + \lambda $u_2$ + \mu $v_2$ \\ z = $z_0$ + \lambda $u_3$ + \mu $v_3$ \end{array}\right\)
\newline Where $u_j$ \in \Re\) and $v_j$ \in \Re\) are fixed constants, form a plane in \Re^3\), containing the point with coordinates \math ($x_0$,$y_0$,$z_0$)\)
\\
\\
Parameters \lamba\) and \mu\) can be used as coordinates in the plane \math (\lambda,\mu) \in \Re^2
\newline ax + by + cz + d = 0\), where \math (a,b,c) \neq (0,0,0)\) defines planes in \Re^3\)
\\
\\
set of points with cartesian \math (x,y,z)\) defined as:
\left\{ \begin{array}{rcl}\math x = $x_0$+\lambda $u_1$ & \mbox{with} & \lambda \in \Re \\ y = $y_0$ + \lambda $u_2$ \\ z = $z_0$ + \lambda $u_3$ \end{array}\right\)
\newline Where $u_j$ \in \Re\) are fixed constants, form a line in \Re^3, passing through the point with coordinates \math($x_0$,$y_0$,$z_0$)\)
\\
\\
Intersection of 2 non-parallel planes as the sets of points fulfilling: \math
\newline $a_1$x + $b_1$y + $c_1$z + $d_1$ = 0
\newline $a_2$x + $b_2$y + $c_2$z + $d_2$ = 0\)
\newline Where \math ($a_j$,$b_j$,$c_j$) \neq (0,0,0)\) and no such real number \alpha\) such that \math ($a_1$,$b_1$,$c_1$) = \alpha($a_2$,$b_2$,$c_2$)\), define lines in \Re^3\)
\\
\\
Given a point (focus) and a line not containing this point (directrix), set of points for which the ratio of the distances to the focus and the directrix is e \in \Re^>\) (eccentricity)

- ellipse when \math e < 1\)

- parabola when \math e = 1\)

- hyperbola when \math e >1\)

Special case for which \math e = 0\) (directrix at infinity) is a circle
\\
\\
Cartesian equation of conic section: \math Ax^2 + Bxy + Cy^2 + Dx + Ey + F = 0\), where A,B,C,D,E,F \in \Re  \triangle := B^2 - 4AC\)

- ellipse for \triangle < 0\)

- parabola for \triangle = 0\)

- hyperbola for \triangle > 0\)
\\
\\
Polar equation of conic section with one focus at the pole: \math r = 1/(1+e cos(\phi + $\phi_0$))\) where \math e,l \in \Re e\) is eccentricity, \math l\) is semi-latus rectus (length of straight line between focus and conic section along direction parallel to directrix)
\\
\\
Circular cone: \math x^2 + y^2 - k^2z^2 = 0\)
\\
\\
Kepler's first law: The planets orbit around the sun on elliptical orbits with the sun in one of the focal points.
\newline Kepler's second law: The line connecting the sun and the sun and the planet sweeps equal areas of space during equal time intervals.
\newline Kepler's third law: The squares of the orbital periods of the planets are proportional to the cubes of the semi-major axes of their orbits.
\\
\\
Position: \math (x,y,z) \in \Re^3\)
\newline Velocity: \math ($v_x$,$v_y$,$v_z$) = (\.{x},\.{y},\.{z})\)
\newline Momentum: \math $p_j$ = m$v_j$
\\
\\
Newton's second law in vector form: \.{p} = F\), where F is external force on "body"/"particle"
\newline Solution \math (x(t),y(t),z(t))\) is called trajectory of particle
\\
\\
Any complex number \math z = x + iy\) (with \math x,y \in \Re\)) can be expressed as \math z = re^i\theta\), with \math r \in \Re^>\) and \phi \in \Re
\newline r = \vert z \vert, \vert z \vert = \math sqrt(x^2+y^2)\) (modulus of z)
\newline \phi = -i ln(z/\vert z \vert)\) (argument or phase of z)
\\
\\
\math $z_1$$z_2$ = $r_1$$r_2$e^i($\phi_1$+$\phi_2$)\)
\newline Multiplying a complex number with unit length $r_2$ = 1 thus amounts to a rotation in the plane around the origin by an angle $\phi_2$ \)

- two rotations amount to one
- rotations are commutative
\\
\\
\math i^2 = j^2 = k^2 = ijk = -1 \)
\newline A quaternion q is then an object of the form \math q = $q_0$+i$q_1$+j$q_2$+k$q_3$\) with \math $q_0,1,2,3$ \in \Re\)
\\
\\
Any point U in \Re^3\) with cartesian coordinates \math ($u_1$,$u_2$,$u_3$) \in \Re^3\) as representing a vector, that points from the origin of the coordinate system to the point U.
\newline Alternate interpretation: translation that maps an arbitrary point \math (x,y,z)\) to \math (x+$u_1$,y+$u_2$,z+$u_3$)\)
\\
\\
Individual real numbers \alpha \in \Re\) are called scalars
\\
\\
\vert u \vert = sqrt($u_1$^2,$u_2$^2,$u_3$^2)\) (euclidean norm)
\newline "normalised", \vert u \vert = 1 = \hat{u}\), unit vector
\newline \math(0,0,0)\) is called a zero vector
\\
\\
vector \math u = ($u_1$,$u_2$,$u_3$) \in \Re^3\) and real number \lambda\) (a scalar), scalar multiplication
\newline \math \lambda u = (\lambda$u_1$,\lambda$u_2$,l\ambda$u_3$)
\newline \therefore \vert \lambda u \vert = \vert \lambda \vert \vert u \vert\) (associative and commutative)

- \lambda u\) traces out a straight line passing through origin and point u
- if \lambda = -1\), we write \math -u = (-$u_1$,...,-$u_n$)\) exact inverse of translation by u
- with 1/\vert u \vert = \lambda, 1/\vert u \vert u = \hat{u}\)
- with \lambda = 0, 0u = 0\) turns into zero vector
\\
\\
Addition and subtraction is done component wise, commutative, associative and distributive
\\
\\
\vert u+v \vert^2 + \vert u-v \vert^2 = 2\vert u\vert^2 + 2\vert v \vert^2\)
\\
\\
Let $u_j$ \in \Re^3\) (with \math j \in $\mathb{N}$\) and \math j < N\)) be vectors and $a_j$ be scalars. The vector:
\newline \math v = \sum_{j=0}^{N} $a_j$$u_j$ \)
\newline Is a linear combination of the vectors $u_j$. $a_j$ are the coefficients.
\\
\\
A set of N vectors {$u_j$} is called linearly independent if no non-trivial linear combination (not all coefficients are 0) of them sums to 0. On the contrary, called linearly dependent if at least one of the vectors in the set as a linear combination of the others: \math $u_k$ = \sum_{j \neq k}$a_j$$u_j$\) (n vectors going in different directions are able to go to any point in n-dimensions)
\\
\\
A set of three vectors a, b, c is a basis of \Re^3\), if any vector \math u \in \Re^3\) can be expressed as a unique linear superposition of a, b and c:
\newline \math u = $u_a$a+$u_b$b+$u_c$c\)

- represents a vector by list of its coefficients in a given basis
- any set of 3 linearly independent vectors $a_j$ in \Re^3\) is a basis of \Re^3\)
\\
\\
Standard basis of \Re^3: (1,0,0),(0,1,0),(0,0,1) \hat{$e_1$} \hat{$e_2$} \hat{$e_3$}\)
\\
\\
Det: (\Re^n\))^n \rightarrow \Re\), fulfils following properties:

- linearity in each argument
- antisymmetry with respect to interchange of arguments
\newline To make determinant uniquely defined, pick a basis {$u_j$} and set det($u_1$,...,$u_n$) = 1
\newline (THINK ABOUT THE POSITIVE CYCLE OF 1 -> 2 -> 3 -> 1 TO DETERMINE SIGNS)
\newline \math det(u,v) = $u_1$$v_2$-$u_2$$v_1$
\newline det(u,v,w) = $u_1$$v_2$$w_3$+$u_2$$v_3$$w_1$+$u_3$$v_1$$w_2$-$u_3$$v_2$$w_1$-$u_2$$v_1$$w_3$-$u_1$$v_3$$w_2$\)
\\
\\
Determinant of n non-zero vectors is zero if and only if the vectors are linearly dependent. Proof:
\newline If 2 vectors are linearly dependent:

\math det(u,v) = \alpha det(v,v) = 0 \)
\newline suppose neither u or v are 0 and are linearly independent, then:

\newline \math \hat{$e_1$} = $\alpha_1$u+$\beta_1$v\), and \hat{$e_2$} = $\alpha_2$u+$\beta_2$v\) where \math det(u,v) = 0\)
\newline Using bi-linearity of determinant:

\math det(\hat{$e_1$},\hat{$e_2$}) = $\alpha_1$$\alpha_2$det(u,u)+$\alpha_1$$\beta_2$det(u,v)+$\alpha_2$$\beta_1$det(v,u)+$\beta_1$$\beta_2$det(v,v)
= 0\)
\newline Which is a contradiction since \math det(\hat{$e_1$},\hat{$e_2$}) = 1\) so if \math det(u,v) = 0\), then u, v are linearly dependent
\\
\\
Let \math u = \alpha v+\beta w

det(u,v,w) = \alpha det(v,v,w)+\beta det(w,v,w)=0\)
\\
\\
Dot product: \math u \cdot v = \vert u \vert \vert v \vert cos(\phi) = $u_1$$v_1$+$u_2$$v_2$+$u_3$$v_3$\)
\newline a) \math u \cdot v = v \cdot u\) (symmetry)
\newline b) \math (\lambda u+ \mu w) \cdot v = \lambda u \cdot v+\mu w \cdot v\) (linearity in first argument)
\newline c) \math u \cdot (\lambda v + \mu w) = \lambda u \cdot v + \mu u \cdot w\) (linearity in second argument)
\newline d) \math u \cdot u \geq 0\) and \math u \cdot u=0 \Leftrightarrow u=0\)
\\
\\
Cauchy-Schwarz Inequality: let u and v be 2 vectors in \Re^3\), then:

\math \vert u \cdot v \vert \leq \vert u \vert \vert v \vert \)
\newline With equality when u and v are linearly dependent
\\
\\
Triangle Inequality: let u and v be 2 vectors in \Re^3\), then:

\math \vert u+v \vert \leq \vert u \vert + \vert v \vert\)
\\
\\
Let u and v be vectors in \Re^3\) with v \neq 0\), there is a unique real \lambda\) such that \math u-\lambda v\) is perpendicular to v: \math u = \lambda v + (u-\lambda v)

\lambda = (u \cdot v)/(\vert v \vert^2)

$proj_v$u = (u \cdot v)/(\vert v \vert^2) v\)
\\
\\
A set of vectors {$u_j$} is said to be orthonormal if each vector is normalised and pairwise orthogonal

\math $u_j$ \cdot $u_k$ = $\delta_jk$ \)
\newline Kronecker symbol $\delta_jk$ = \left\{ \begin{arrary}{rcl} 1 & if j=k \\ 0 & otherwise \end{array}\right\)
\\
\\
Basis composed of orthonormal vectors is an orthonormal basis

- set of three orthonormal vectors in \Re^3\) forms a basis
- if {$w_1$,$w_2$,$w_3$} is an orthonormal basis of \Re^3\) and \math u = $\alpha_1$$w_1$+$\alpha_2$$w_2$+$\alpha_3$$w_3$\) then \math $a_j$ = u \cdot $w_j$\)
\\
\\
Given 2 vectors u and v in \Re^3\) and orthonormal basis {$b_j$}
\newline \math u = $u_1$$b_1$+$u_2$$b_2$+$u_3$$b_3$\) and \math v = $v_1$$b_1$+$v_2$$b_2$+$v_3$$b_3$\)
\newline Their inner product is given by \math u \cdot v = $u_1$$v_1$+$u_2$$v_2$+$u_3$$v_3$\)
\\
\\
\math u = \alpha $\hat{e}_1$+\beta $\hat{e}_2$, v = -\beta$\hat{e}_1$+\alpha $\hat{e}_2$\) orthogonal to \^{u}, \lambda \^{v}\) vectors in \Re^2\) orthogonal to \^{u}, \lambda \in \Re\)
\\
\\
Plane P oriented by orthonormal basis {$\hat{e}_1$, $\hat{e}_2$}, an orthonormal basis {\^{u},\^{v}\)} is said to be right-handed if vectors are given by:
\newline \^{u} = \alpha $\hat{e}_1$+\beta $\hat{e}_2$\) and \^{v} = -\beta $\hat{e}_1$+\alpha $\hat{e}_2$\)
\newline and left_handed by: \^{u} = \alpha $\hat{e}_1$+\beta $\hat{e}_2$\) and \^{v} = \beta $\hat{e}_1$-\alpha $\hat{e}_2$\)
\newline with \math (\alpha,\beta) \in \Re^2\) and \alpha^2+\beta^2=1
\\
\\
det(u,v) = \vert u \vert \vert v \vert sin(\phi) \)
\\
\\
Consider 2 linearly independent vectors $u_1$ and $u_2$ in \Re^3\)

- There exists a vector v orthogonal to both $u_1$ and $u_2$

- Vectors orthogonal to v are the linear combinations of vectors $u_1$ and $u_2$
\\
\\
Vector products of $u_1$ and $u_2$ in \Re^3\) denoted by $u_1$x$u_2$:
\newline \math (($y_1$$z_2$-$y_2$$z_1$),($z_1$$x_2$-$z_2$$x_1$),($x_1$$y_2$-$x_2$$y_1$))\)

- Vector product 0 if and only if $u_1$ and $u_2$ are collinear

- Vector product is orthogonal to both $u_1$ and $u_2$

- When $u_1$ and $u_2$ are not collinear, vector product is proportional to the unique vector orthogonal to $u_1$ and $u_2$

- Antisymmetric bilinear form

a) \math u x v = -v x u\) (antisymmetry)

b) \math (\lambda u + \mu w) x v = \lambda u x v + \mu w x v\) (linearity in first)

c) \math u x (\lambda v + \mu w) = \lambda u x v + \mu u x w\) (linearity in second)
\\
\\
Let u and v be vectors in \Re^3\), \vert u x v \vert^2 = \vert u \vert^2 \vert v \vert^2 - (u \cdot v)^2
\newline \vert u x v \vert = \vert u \vert \vert v \vert sin(\theta)\)
\\
\\
Let {\^{u},\^{v},\^{w}} an orthonormal basis: right-handed if \^{w} = \^{u}x\^{v} and left-handed if \^{w} = -\^{u}x\^{v}
\newline (first finger: u, second finger: v, thumb: u x v)
\\
\\
Scalar triple product \math [u,v,w] = u \cdot (v x w) = det(u,v,w)\)
\newline volume of parallelepiped \vert [u,v,w] \vert
\newline [u,v,w] = [v,w,u] = [w,u,v] = -[u,w,v] = -[w,v,u] = -[v,u,w] \)
\\
\\
Vector triple product \math u x (v x w)
\newline u x (v x w) = (u \cdot w)v - (u \cdot v)w\)

\section{Logic and Sets}

Logical proposition is a true/false statement
\newline "and" \wedge
\newlilne "or" \vee
\newlline "not" \neg
\newline "implies" \Rightarrow (Only way to disprove is when P is true and Q is false)
\newlline "if and only if" \Leftrightarrow (P\Rightarrow Q) \wedge (Q \Rightarrow P), P = Q
\\
\\
Propositional formula is smallest set off strings of propositions and connectives such that, if P and Q are propositional formula then so are P \wedge Q, P \vee Q, \neg P, Q \Rightarrow P, P \Leftrightarrow Q
\newline Logically equivalent if they have the same truth table
\\
\\
Law of excluded middle: \neg (\neg P) \Leftrightarrow P
\newlilne De Morgan's laws: \neg (P \vee Q) = (\neg P) \wedge (\neg Q), \neg (P \wedge Q) = (\neg P) \vee (\neg Q)
\newline Transitivity: we know P \Rightarrow Q and Q \Rightarrow P, deduce P \Rightarrow R
\newline Contrappositive: (P \Rightarrow Q) \Rightarrow (\neg Q \Rightarrow \neg P)
\newline Distributivity: P \wedge (Q \vee R) \Leftrightarrow (P \wedge Q) \vee (P \wedge R), P \vee (Q \wedge R) \Leftrightarrow (P \vee Q) \wedge (P \vee R)
\\
\\
Sets {}

- Only have elements once

- Don't come in a particular order

- Empty set denoted as \varnothing

- x \in\) X, "x is an element of X"

- Subsets \subseteq

- X \cup Y = {t \vert t \in X \vee t \in Y}

- X \cap Y = {t \vert t \in X \wedge t \in Y}

- X \diagdown Y, {x \in X \vert x \notin Y} \)
\\
\\
Complement \={X} = {t \in \Omega \vert t \notin X}\) were X \subseteq \Omega \)
\newlline Let X \subseteq \Omega\) and Y \subseteq \Omega\) aer subsets:
\newline \={X \cup Y} = \={X} \cap \={Y}
\newline \={X \cap Y} = \={X} \cup \={Y}\)
\\
\\
\forall "for all", \exists "there exists"
\\
\\
$\bigcup_{n=0}^{\infty} $X_n$$ defined as {x \in \Omega \vert \exists x \in N, x \in $X_n$}
\newline $\bigcap_{n=0}^{\infty} $X_n$$ defined as {x \in \Omega \vert \forall x \in N, x \in $X_n$}
\\
\\
\exists x \in \varnothing\), P is always false (if P were true, would imply such an x exists but it doesn't)
\newline \forall x \in \varnothing\), P is always true (it is the logical negation, \exists x \in \varnothing, \neg P\))
\\
\\
\math f:X \rightarrow Y\) is a function if: (domain and codomain)
\newline \forall x \in X \exists y \in Y (f(x) = y \wedge \forall z \in Y (f(x) = z \Rightarrow y = z)

- When you give x \in X, it will return an element

- Function depends on domain and codomain

- Value of f(x) can't change over time

- Cannot e one-to-many
\\
\\
Image/Range of f: {y \in Y \vert \exists x \in X, f(x) = y}
\newline \mapsto  "maps to"
\\
\\
Identity function: if x \in X, x \mapsto  x
\newline Empty function: function from empty set X to Y
\\
\\
\math f:A \rightarrow B and g:B \rightarrow C\)), then \math g \circ f: A \rightarrow C\), (g composed with f)
\newline \math $f_1$: X \rightarrow Y\) and \math $f_2$: X \rightarrow Y\), tey are equal if, \forall x \in X\) we have \math $f_1$(x ) = $f_2$(x )\)
\\
\\
Injective: if \forall a,b \in X, f(a) = f(b) \Rightarrow a=b
\newline Surjective: if \forall y \in Y, \exists x \in X, f(x) = y
\newline Bijective: if it is both injective and surjective
\\
\\
If f and g are both injective/surjective/bijective, then so is g \circ f
\\
\\
Inverse: say \math f:X \rightarrow Y\) is a function, \math g:Y \rightarrow X\) is a two-sided inverse of \math f\)) if:
\newline \forall x \in X, g(f(x)) = x
\newline \forall y \in Y, f(g(y)) = y
\newline \math f:X \rightarrow Y\) has a two-sided inverse if and only if it's a bijection, proof:
\\
\\
Proving injectivity of f:

Say p \in X and q \in X, and f(p) = f(q)
\newline Applying inverse function g(f(p)) = g(f(q))
\newline g(f(q)) = p
\newline g(f(p)) = q, by definition
\newline hence p = q
\\
\\
Proving surjectivity of f:

Say y \in Y, define x = g(y)
\newline f(x) = f(g(y)) = y by definition
\newline hence there exists x \in X with f(x) = y
\\
\\
Proving other way:

Say y \in Y, we would like to define an element g(y) \in X by bijectivity (we know that tere exists x \in X st. f(x) = y (surjectivity), due to injectivity, only one such x exists), we define g(y) to be the element x \in X with f(x) = y, unique. Therefore f(g(y)) = y.

Say x \in X, define y = f(x), since f(g(y)) = y, g(y) is the only element in x such tat tis holds, tereffore g(y) = x, hence g(f(x)) = x.
\\
\\
A binary relation R on a set X is a subset of X^2, this set is denoted by R(x,y)
\\
\\
Sets as function: as a fffunction from \Re^2\) to prop {true,false}
\newline Functions as sets: f:X \rightarrow Y is a subset of X x Y, {(x,y) \in X x Y \vert f(x) = y}
\\
\\
Predicate: a property - proposition attached to each element of a set
\\
\\
Reflexive: \forall x \in X, R(x,x) is true
\newline Symmetric: \forall a,bb \in X, R(a,b) \Rightarrow R(b,a)
\newlline Antisymmetric: \forall a,b \in X, R(a,b) \wedge R(b,a) \Rightarrow a = b
\newline Transitive: for every a,b,c \in X, we have (R(a,b) \wedge R(b,c)) \Rightarrow R(a,c)

- R is a partial order if it's reflexive, antisymmetric and transitive
- R is total if for all a,b \in X, either R(a,b) \vee R(b,a) is true
- R is a total order if it's a partial order and total
\\
\\
R is an equivalence relation if it is reflexive, antisymmetric and transitive
\\
\\
Let X be a set and let ~ be an equivalence relation on X. Let s \in X bbe an arbitrary element We define the equivallence class of S, written cl(s), to be the set of elements of X which s is related to. 
\newline cl(s) = {x \in X: s~x}

\end{document}
